\section{Язык теории множеств. Равенство множеств. Аксиомы равенства, пары, объединения, степени, выделения. Упорядоченные пары, декартово произведение, бинарные отношения между множествами, отношение эквивалентности, фактормножество, функции}

\definition{Основными неопределяемыми понятиями теории множеств являются понятие \textbf{множества} и понятие \textbf{быть элементом} множества. Неформально, множество понимается как некоторая (конечная или бесконечная) совокупность объектов, рассматриваемая как единое целое, отдельный объект. Объекты, входящие в совокупность, называются \textbf{элементами} данного множества. Запись $x \in A$ означает, что $x$ есть элемент множества $A$, или $x$ принадлежит $A$. Два множества считаются равными, то есть совпадают, если у них одни и те же элементы (иногда это определение называют \textbf{аксиомой объемности}):

$$ x = y \overset{\underset{\mathrm{def}}{}}{\Longleftrightarrow} \forall z ( z \in x \leftrightarrow z \in y ) $$

\subsection{Некоторые аксиомы теории множеств Цермело–Френкеля}

\begin{enumerate}

\item{\textbf{Аксиома равенства}: Равные множества $x$ и $y$ являются элементами одних и тех же множеств.

$$ x = y \rightarrow \forall z ( x \in z \leftrightarrow y \in z ) $$}

\item{\textbf{Аксиома пары}: Для любых $x$ и $y$ найдется множество $ z = {x, y}$, элементами которого являются в точности $x$ и $y$.

$$ \forall x, y \exists z : \forall u (u \in z \leftrightarrow ( u = x \lor u = y)) $$}

\item{\textbf{Аксиома объединения}: Для любого множества $X$ существует множество $Y = \bigcup X$, содержащее в точности те элементы, которые принадлежат хотя бы одному из элементов множества $X$.

$$ \forall x \exists y: \forall u ( u \in y \leftrightarrow \exists z (u \in z \land z \in x)) $$}

\item{\textbf{Аксиома степени}: Для любого $X$ существует множество $ Y = \mathcal{P} (X)$ всех подмножеств $X$.

$$ \forall x \exists y : \forall z ( z \in y \leftrightarrow z \in x) $$}

\item{\textbf{Схема аксиом выделения}: Для любого свойства $\phi (x)$ и множества $X$ найдется множество $Y = \{x \in X: \phi (x)\}$, содержащее те и только те элементы $x \in X$, которые удовлетворяют свойству $\phi$.

$$ \forall x \exists y: \forall u ( u \in y \leftrightarrow (u \in x \land \phi (u))) $$}

\end{enumerate}

\subsection{Другие определения}

\definition{\textbf{Упорядоченная пара} -- это сопоставление паре множеств $x$, $y$ некоторое множество $z$, обозначаемое $\langle x, y \rangle$, таким образом, чтобы для всех $x_1, x_2, y_1, y_2$

$$ \langle x_1, y_1 \rangle = \langle x_2, y_2 \rangle \Longleftrightarrow (x_1 = x_2 \land y_1 = y_2) $$

Один из способов задания (по Куратовскому):  $\langle x, y \rangle \overset{\underset{\mathrm{def}}{}}{=} \{\{x, y\}, \{x\}\}$}

\definition{Множество всех упорядоченных пар элементов множеств $A$ и $B$ называется \textbf{декартовым произведением} $A$ и $B$:

$$ A \times B = \{\langle x, y \rangle : x \in A \land y \in B\} $$

Заметим, что $\langle x, y \rangle \in \mathcal{P} ( \mathcal{P} ( A \cup B))$, которое существует по аксиомам объединения и степени, а значит произведение $ A \times B$ существует по аксиоме выделения.}

\definition{\textbf{Бинарным отношением} между множествами $A$ и $B$ называется любое подмножество $R \subset A \times B$. Если $A = B$, то говорят о бинарном отношении \textbf{на} множестве $A$. Вместо $\langle x, y \rangle \in R$ часто пишут $xRy$. Бинарное отношение $R$ на множестве $A$ называется

$\bullet$ \textbf{рефлексивным}, если $ \forall x \in A$ $xRy$;

$\bullet$ \textbf{симметричным}, если $\forall x,y \in A (xRy \rightarrow yRx)$;

$\bullet$ \textbf{транзитивным}, если $\forall x,y,z \in A (xRy \land yRz \rightarrow xRz)$.

Отношение, обладающее всеми тремя этими свойствами называется \textbf{отношением эквивалентности}.}

\definition{Множество всех классов эквивалентности $A$ по отношению $R$ называется \textbf{фактормножеством} и обозначается $A / R$:

$$ A \ R = \{ x_R : x \in A \} $$}

\subsection{Функции}
\definition{Отношение $R \subset A \times B$ называется

$\bullet$ \textbf{тотальным}, если $ \forall x \in A$ $\exists y \in B$ $xRy$;

$\bullet$ \textbf{сюръективным}, если $ \forall y \in B$ $\exists x \in A$ $xRy$;

$\bullet$ \textbf{функциональным}, если $ \forall x \in A$ $\forall y_1, y_2 \in B (xRy_1 \land xRy_2 \rightarrow y_1 = y_2)$;

$\bullet$ \textbf{инъективным}, если $ \forall y \in B$ $\forall x_1, x_2 \in A (x_1 Ry \land x_2 Ry \rightarrow x_1 = x_2)$.

\textbf{Функцией} $f$ из $A$ в $B$ называется тотальное и функциональное бинарное отношение между $A$ и $B$ (обозначение $f : A \rightarrow B$). Слово \textbf{отображение} есть синоним слова функция. Если $f$ -- функция, то мы пишем $f (x) = y$ вместо $\langle x, y \rangle \in f$.}
